\documentclass[a4paper,11pt]{article}
\usepackage[utf8]{inputenc}
\usepackage[letterpaper,margin=0.5in]{geometry}
\usepackage[T1]{fontenc}
\usepackage{tgbonum}
\usepackage{csquotes}
\usepackage{enumitem}
\usepackage{parskip}
\usepackage{titlesec}
\usepackage{hyphenat}
\usepackage{fancyhdr}
\usepackage{tabularx}
\usepackage{multirow}
\usepackage{authblk}
\usepackage[dvipsnames]{xcolor}

\setlength{\columnsep}{2em}
\setcounter{secnumdepth}{4}

\pagestyle{fancy}
\fancyhead{}
\fancyfoot{}
\fancyfoot[R]{Page \thepage}

\titlespacing{\paragraph}{0pt}{1em}{\medskipamount}

\title{%
  Variants for GDW's Imperium  \\
  \vspace{0.5em}
}

\author{Daniel J. Berger}

\begin{document}
\maketitle

The following changes are options designed to address several key issues which we found to be problematic to our enjoyment of the game. No attempt is made to conform to the fiction. These are strictly designed specifically to make the game more interesting from a play balance point of view.

\section{Sequence of Play}

Instead of the Imperial and Terran players conducting Maintenance and Production and the start of their respective turns, both the Terran and Imperial player perform maintenance and Production at the start of each turn, Imperial player first.
\hfill

\textit{Rationale}: The current system creates a strange dynamic where each side plops down their reinforcements and can then bludgeon their opponent before they've had a chance to counter it. The end result is a sort of weird brawl where each side is throwing a knockout blow with a massed fleet but neither can really dodge or counter it.

\section{Special Attacks}

You may not use \textit{High Intensity Missile Fire} (HIMF) when your opponent terminates combat.

\textit{Rationale:} There's no real downside to it if you're not assaulting a planet, and it feels too powerful. You could also argue that the effectiveness of HIMF is offset by the (presumed) increase in range as the enemy retreats.

If you perform a \textit{Suicide Attack} when your opponent terminates combat, the targeted ships may fire at the attacking ships first. This is an exception to the rule that ships terminating combat may not fire.

\textit{Rationale:} I think it's fair to say that even in retreat they would be able to fire at an approaching ship.

\section{Capital Ships}

Any ship that costs at least 10 RP (i.e. Strike Cruiser, Heavy Cruiser, Battleship and all Dreadnoughts) suffers a disruption result the first time they are hit rather than be destroyed. If a disrupted capital ship is hit again (or was already disrupted at the start of space combat or planetary defense) then it is destroyed.

\textit{Rationale:} Currently the capital ships are essentially useless. They are too expensive, too brittle, have a high maintenance cost and are unlikely to survive between wars. The upshot is that no one builds them, opting for large numbers of smaller units instead. This should also make capital ships more valuable in assaults since they can withstand a hit.

\section{Scouts}

During the preparation step of combat you may "attach" a scout to a ship that has already been placed. The scout is considered to be providing ECM to the attached ship. Enemy ships firing at a ship with an attached scout suffer a -1 penalty to the die roll. This applies to both beam and missile combat.

\textit{Restrictions:} A scout may not attach to a fighter, monitor, or another scout. No more than one scout may be attached to a given ship.

An attached scout is considered screened and may not be targeted. However, it may not fire during combat.

\textit{Rationale:} Currently the scouts are not very useful, especially the Terran scouts. Currently their primary purpose is to be speed bumps or swarm enemy fleets with dummy targets. This gives them a mission and reason to live.

\section{Destroyers}

Destroyers fire at +1 against fighter units.

\textit{Rationale:} Makes destroyers slightly more interesting and worth building.

\section{Appeals to the Emperor}

The Imperial player does not lose a Glory point on a roll of 5 or 6 (no result) when making an Appeal to the Emperor. Instead, the penalty is that the Imperial player may not make another appeal on the following turn.

\textit{Rationale:} Failing appeals can be decisively bad for the Imperial player given how coarse the victory point system is. This variant reduces the bite, only penalizing the Imperial player on very bad rolls. Thus, there is still some risk.
\end{document}
