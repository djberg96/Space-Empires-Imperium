\documentclass[a4paper,11pt]{article}
\usepackage[utf8]{inputenc}
\usepackage[letterpaper,margin=0.5in]{geometry}
\usepackage[T1]{fontenc}
\usepackage{tgbonum}
\usepackage{csquotes}
\usepackage{enumitem}
\usepackage{parskip}
\usepackage{titlesec}
\usepackage{hyphenat}
\usepackage{fancyhdr}
\usepackage{tabularx}
\usepackage{multirow}
\usepackage{authblk}
\usepackage[dvipsnames]{xcolor}

\setlength{\columnsep}{2em}
\setcounter{secnumdepth}{4}

\pagestyle{fancy}
\fancyhead{}
\fancyfoot{}
\fancyfoot[R]{Page \thepage}

\titlespacing{\paragraph}{0pt}{1em}{\medskipamount}

\title{%
  Variants for GDW's Imperium  \\
  \vspace{0.5em}
  \date{\today\\v1.1}
}

\author{Daniel J. Berger}

\begin{document}
\maketitle

The following changes are options designed to address several key issues which we found to be problematic to our enjoyment of the game. No attempt is made to conform to the fiction. These are strictly designed specifically to make the game more interesting from a play balance point of view.

\section{Sequence of Play}

Instead of the Imperial and Terran players conducting Maintenance and Production and the start of their respective turns, both the Terran and Imperial player perform maintenance and Production at the start of each turn, including Imperial Intervention and any Appeals to the Emperor.

In the first war the Imperial player conducts maintenance and production first. In subsequent wars, the winner of the previous war goes first.

\textit{Rationale}: The current system creates a strange dynamic where each side plops down their reinforcements and can then bludgeon their opponent before they've had a chance to counter it. The end result is sort of like \textit{Risk} after you've traded in a set of cards. You bludgeon your opponents, and then they do the same thing, and the cycle repeats.

\section{Special Attacks}

\subsection{High Intensity Missile Fire}

You may not use \textit{High Intensity Missile Fire} (HIMF) when your opponent terminates combat.

\textit{Rationale:} There's no real downside to it if you're not assaulting a planet, and it feels too powerful. You could also argue that the effectiveness of HIMF is offset by the (presumed) increase in range as the enemy retreats.

\subsection{Suicide Attack}

If you perform a \textit{Suicide Attack} when your opponent terminates combat, the targeted ships may fire at the attacking ships first. This is an exception to the rule that ships terminating combat may not fire.

\textit{Rationale:} I think it's fair to say that even in retreat they would be able to fire at an approaching ship.

\subsection{Short Range Missile Fire}

Add 1 to the die roll when using \textit{Short Range Missile Fire} (SRMF).

\textit{Rationale:} The idea here is to give a slight buff to something that only really only matters to a few ship types in particular. The basis for this variant thematically is that anti-missile systems (abstracted generically into "screens") would have a harder time countering missiles at close range.

\section{Capital Ships}

Any ship that costs at least 10 RU (i.e. Strike Cruiser, Heavy Cruiser, Battleships and Dreadnoughts) suffers a disruption result the first time they are hit rather than be destroyed. If a disrupted capital ship is hit again (or was already disrupted at the start of space combat or planetary defense) then it is destroyed.

\textit{Rationale:} Currently the capital ships are essentially useless. They are too expensive, too brittle, have a high maintenance cost and are unlikely to survive between wars. The upshot is that no one builds them, opting for large numbers of smaller units instead. This should also make capital ships more valuable in planetary assaults since they can withstand a hit.

\section{Scouts}

During the preparation step of combat you may "attach" a scout to a ship that has already been placed. Place the scout and the ship that it's attached to at the same time. The scout is considered to be providing ECM to the attached ship. Enemy ships firing at a ship with an attached scout suffer a -1 penalty to the die roll. This applies to both beam and missile combat.

\textit{Restrictions:} A scout may not attach to a fighter, monitor, or another scout. No more than one scout may be attached to a given ship. A disrupted scout may not use this ability.

An attached scout is considered screened, may not be targeted, and may not fire. However, if the ship it was attached to is destroyed then it is no longer considered screened. Any enemy ships that were lined up against the destroyed ship that have not yet fired may instead fire at the attached scout.

\textit{Rationale:} Currently the scouts are not very useful, especially the Terran scouts. Currently their primary purpose is to be speed bumps or swarm enemy fleets with dummy targets. This gives them a mission and reason to live.

\section{Destroyers}

Destroyers (DD) fire at +1 against fighter units.

\textit{Rationale:} Makes destroyers slightly more interesting and more useful against fighter swarms.

\section{Light Cruisers}

Light Cruisers (CL) may carry a 1 or 2 strength infantry (regular or jump troop) in the same manner as a transport.

\textit{Rationale:} Makes light cruisers slightly more interesting, and gives players a small upside to drawing low strength infantry.

\section{Fighters}

A maximum of 3 fighters may operate at an Outpost. If you have more than 3 fighters at an Outpost at any point, they are considered to be on the surface.

Any number of fighters may operate at a World.

\textit{Rationale:} This is mainly meant to defeat massed stacks of fighters at Outposts and it feels thematic.

\section{Ground Combat}

If, after a round of ground combat, one side has troops and the other side has only non-troop counters remaining, the player with only non-troops loses the ground combat immediately. All of that player's remaining counters (including Outpost markers) are eliminated, except World markers which are instead neutralized.

\textbf{Exception:} Treat a defending PD unit as a troop counter for purposes of this rule. It still only defends, but its presence protects the other non-troop counters from elimination. If a non-troop counter is selected during placement then a PD unit must be selected before other non-troop counters.

\textit{Rationale:} The rules as written state that there is only ever one round of combat per Surface sub-phase, stating \textit{"A planetary surface box cannot be conquered in one sub-phase if the number of attacking troops is less than the number of defending counters"}. However, this quickly leads to nonsensical tactics where the players (probably the Terrans) can simply land hordes of cheap space units (transports, fighters, scouts, etc) on the surface that make conquest of any planet virtually impossible given the counter mix limits and time constraints. This change makes more thematic sense, is more consistent from a rules perspective (currently only the attacker is eliminated if he has no troops) and also makes troops (and to a lesser extent PD units) more valuable.

\section{Appeals to the Emperor}

The Imperial player does not lose a Glory point on a roll of 5 or 6 (no result) when making an Appeal to the Emperor. Instead, the penalty is that the Imperial player may not make another appeal on the following turn.

\textit{Rationale:} Failing appeals can be decisively bad for the Imperial player given how coarse the victory point system is. This variant reduces the bite, only penalizing the Imperial player on very bad rolls. Thus, there is still some risk, but less than before.
\end{document}
